\documentclass[final,leqno]{siamltex1213}

%\usepackage{inconsolata}

\usepackage{fontspec}

\usepackage{newunicodechar}
\newunicodechar{α}{\ensuremath \alpha}
\newunicodechar{β}{\ensuremath \beta}

\usepackage{listings}
\lstset{basicstyle=\ttfamily, numbers=left, numberstyle=\tiny, mathescape=true}
\usepackage{todonotes}
\usepackage{algorithm}
%\usepackage{algorithmic}
\bibliographystyle{siam}

\title{Practical Lanczos bidiagonalizations in Julia
    \thanks{This
        work was supported by the
	%\todo{ALAN to supply}
	}}

\author{%
    Jiahao Chen
    \thanks{Computer Science and Artificial Intelligence Laboratory,
           Massachusetts Institute of Technology,
           Cambridge, Massachusetts, 02139 ({\tt jiahao@mit.edu})}
    %
    \and
    Andreas Noack
    \thanks{Computer Science and Artificial Intelligence Laboratory,
            Massachusetts Institute of Technology,
            Cambridge, Massachusetts, 02139 ({\tt noack@mit.edu})}
    %
    \and
    Alan Edelman
    \thanks{Department of Mathematics and Computer Science and Artificial Intelligence Laboratory,
            Massachusetts Institute of Technology,
            Cambridge, Massachusetts, 02139 ({\tt edelman@mit.edu})}
}


\begin{document}

\maketitle

\begin{abstract}
We describe the implementation of a practical Krylov-Schur
\end{abstract}

\begin{keywords}
\end{keywords}

\begin{AMS}
\end{AMS}

\pagestyle{myheadings}
\thispagestyle{plain}
\markboth{J. CHEN, A. NOACK AND A. S. EDELMAN}{Practical SVD in Julia}


This paper is to describe the role that partial factorizations play in
organizing the various methods for doing singular value decompositions in
Julia


\section{Introduction}

The singular value decomposition (SVD) of a $m\times n$ matrix is
the product of matrices
\[
A=USV^{T}
\]
where $S$ is a diagonal real matrix, ...

SVDs are popular for tint many contests itncluding the princiapal
components anlayss bkah blah blaha dn it si salso known as the Kahunen-Loeve
decompositoin in other conetxts.

Of particular interest in the use of low rank approximations for SVDS.
In this case it is not necessary to compute $U$, $S$ and $V$ and
their entirety; rather, it suffices to determine the first $k$ diagonal
entries of $S$, being the $k$ largest singular values, and their
corresponding columns of $U$ and $V$.

The Lanczos bidiagonalization~\cite{Golub1965} is a ntural choice.

This paper describes an implementation of the Lanczos bidiagonalization
method written in pure Julia.


\section{Implementing simple bidiagonalization in Julia}

\begin{algorithm}
\caption{Simple Golub-Kahan-Lanczos bidiagonalization in Julia}

\begin{lstlisting}
function svd_gkl(A, q; maxiter=minimum(size(A)))
    m, n = size(A)
    T = eltype(A)
    Tr = typeof(one(T)/norm([one(T)]))

    $α$s = Tr[]
    $β$s = Tr[]
    P = zeros(T, m, 0)
    Q = [q]

    $β$ = zero(Tr)
    p = zeros(size(A, 1))
    for iter in 1:maxiter
        p = A*q - $β$*p
	$α$ = norm(p)
        p = p/$α$
        push!($α$s, α)
        P = [P p]

        q = A'p - $α$*q
        $β$ = norm(q)
        q = q/$β$
        push!($β$s, β)
        Q = [Q q]
    end

    B = Bidiagonal($α$s, $β$s[1:maxiter-1], true)
    F = svdfact(B)
    LinAlg.SVD(P'F[:U], F[:S], F[:Vt]*Q[:,1:maxiter]')
end
\end{lstlisting}
\end{algorithm}

Let's explain some feature of Julia exhibited by this code listing.



\section{Theory}


\subsection{The basic bidiagonalization}

Lanczos bidiagonalization is usually described with the following
algorithm:


\subsection{Implicit restarting}


\subsection{Error analysis}


\section{Results}

Comparison with flashpca or randomized subspace interaction.

\section*{Acknowledgments}
We thank the Julia development community for their contributions of free and
open source software. Julia is free software that can be downloaded from
\url{julialang.org/downloads}. J.C. would also like to thank Alex Townsend
(MIT) and David Silvester (Manchester) for many insightful discussions.

\bibliography{svd}

\end{document}
