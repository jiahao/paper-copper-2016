\documentclass{letter}

\date{June 30, 2016}

\usepackage{color}
\usepackage{fancyhdr}
\usepackage{graphics}
\usepackage{hyperref}
\usepackage{times}
\usepackage[normalem]{ulem}

\usepackage{geometry}
\geometry{tmargin=1in,bmargin=1in,lmargin=1in,rmargin=1in}

\newcommand{\affiliation}{Computer Science and Artificial Intelligence Laboratory}
\newcommand{\room}{32-206}
\author{Jiahao Chen}
\newcommand{\email}{jiahao@mit.edu}
\newcommand{\fax}{(617) 253-7030}
\renewcommand{\telephone}{(617) 953-8640}

%These are the standard color palette defined by MIT
\definecolor{MITGrey}{cmyk}{0,0,0,0.5}
\definecolor{MITRed}{cmyk}{0,1,0.65,0.34}

%These are the standard typefaces defined by MIT
\DeclareFixedFont{\boldfont}{OT1}{cmss}{bx}{n}{8}
\DeclareFixedFont{\regularfont}{OT1}{cmss}{m}{n}{8}
\DeclareFixedFont{\italicfont}{OT1}{cmss}{m}{sl}{8}

%The header is 4 lines long, so adjust header spacing accordingly
\addtolength{\headheight}{28pt}
\addtolength{\textheight}{-28pt}

%Color the dividing lines
\renewcommand{\headrule}{\color{MITRed}{\hrule}}
\renewcommand{\footrule}{\color{MITGrey}{\hrule}}
\renewcommand{\headrulewidth}{0.3mm}
\renewcommand{\footrulewidth}{0.3mm}

%make sure the header goes on the first page

\makeatletter

%Left side of header: Name, affiliation, date, title and pages
\lhead{\textcolor{MITGrey}{
    \textcolor{black}{\boldfont \@author} \\\vspace{-3pt}
    {\regularfont
    Email\hspace{9mm}\email \\\vspace{-3pt}
    Telephone \hspace{2mm}\telephone \\\vspace{-3pt}
    Fax\hspace{11mm}\fax \vspace{-3pt}}
}}

%Center of header - MIT logo
%the bounding box argument is for some latexes that don't know what size it is
\chead{\scalebox{.5}{\includegraphics{mitlogo.pdf}}\vspace{-4pt}\hspace{32pt}}

%Right side of header - physical and email addresses
\rhead{\textcolor{MITGrey}{
    \textcolor{black}{\boldfont \affiliation\\\vspace{-3pt}
    Massachusetts Institute of Technology}\\\vspace{-3pt}
    \regularfont{
      32 Vassar Street, Room \room\\\vspace{-3pt}
      Cambridge, Massachusetts 02139-4309}}}

 \cfoot{}

\renewcommand{\opening}[1]
  {\ifx\@empty\fromaddress
    {\raggedleft\@date\par}%
  \else  % home address
    {\raggedleft\begin{tabular}{l@{}}\ignorespaces
      \fromaddress \\*[2\parskip]%
      \@date \end{tabular}\par}%
  \fi
  \pagestyle{fancy}%
  \vspace{2\parskip}%
  {\raggedright \toname \\ \toaddress \par}%
  \vspace{2\parskip}%
  #1\par\nobreak}

%%% End of fancy MIT header

\makeatother

\begin{document}

\letter{Professor Michele Benzi\\
Guest Editor-in-Charge of the Special Section\\
SIAM Journal on Scientific Computing\\
Society for Industrial and Applied Mathematics\\
3600 Market Street, 6th Floor\\
Philadelphia, Pennsylvania 19104-2688\\
USA}

\opening{Dear Prof. Benzi:}

\signature{%
\vspace{-1cm}\includegraphics{signature.pdf}\\
Jiahao Chen, Ph.D.\\
Research Scientist\\
Massachusetts Institute of Technology\\
Computer Science and Artificial Intelligence Laboratory\\
}
\begin{center}
\textbf{\uline{Article submission for SISC Copper Mountain Special Section}}
\par\end{center}

Please find enclosed our manuscript entitled ``Fast computation of the
principal components of genomics data in Julia''. The manuscript
describes our investigation into the software used in the field of
statistical genetics to compute principal component analyses, and
places the applications of PCA to precision medicine on a firm
statistical and linear algebraic foundation. We also show that
superior performance and accuracy is attained by our own
implementation in the Julia programming language of Lanczos
bidiagonalization.

We believe that this manuscript, which attempts to bridge the
disciplinary gap between applied mathematics and the exciting and
upcoming field of statistical genetics and precision medicine, will be
of general interest to the readership of the \textit{SIAM Journal of
Scientific Computing}.

If permissible, we would like to suggest as potential reviewers Dr.\
Richard Lehoucq at Sandia National Laboratories, Dr.\ Jack Poulson at
Google Research, and Prof.\ Andreas Stathopoulos at the College of
William and Mary, all of whom are experts in iterative methods in
computational linear algebra and would be very familiar with their
applications to principal components analysis.

Once again, we thank you and the organizers of the 14th Copper
Mountain Conference on Iterative Methods for allowing us to
participate as speakers. We hope that you will agree with us that this
manuscript will be deserving of publication in the \textit{SIAM
Journal of Scientific Computing}, and look forward to your favourable
reply.

\closing{Yours faithfully,}

\end{document}
